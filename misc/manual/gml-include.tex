
%%%
\subsection{Introduction to the specification language \GML{}}

Within the \GGL{}, graphs and graph rewrite rules are specified in a
language called \GML{} (Graph Modeling Language). Essentially,
\GML{} is composed of hierarchical \textbf{key--value} pairs. Keys are
usually strings (some identifiers) and values specify the value of the
corresponding key. Values are either single values (numbers, strings, etc) or
lists of key--value pairs. Lists \textbf{must always} be enclosed in opening '['
and closing ']' brackets in \GML{}. Nesting of lists to arbitrary depth is
allowed in \GML{}. The general structure of a \GML{} specification
looks as follows: \startGML
\begin{verbatim}
key1 [
 key2 value2
 key3 [
  key4 value4
  key5 value5
 ]
 key6 value6
] 
\end{verbatim}
\endGML

\noindent In the above code snippet keys 1 and 3 have both a list as value
(hence the brackets). Keys 2, 4--6 are key--value pairs where the
corresponding values are single values such as numbers or strings.

%%%
\subsection{BNF of \GML{}}
% BNF grammar of GML (http://www.infosun.fim.uni-passau.de/Graphlet/GML/)

Following is the grammar specification of \GML{} in \emph{Boyes Normal
Form (BNF)}.

\startGML
\begin{verbatim}
gml       ::= keyvalues
keyvalues ::= keyvalue (keyvalue)*
keyvalue  ::= key value
key       ::= ['a'-'z''A'-'Z']['a'-'z''A'-'Z''0'-'9']
value     ::= real | integer | string | list | operator
real      ::= sign? digit  '.' digit+ mantissa?
integer   ::= sign? digit+
operator  ::= '<' | '=' | '>' | '!'
string    ::= '"' instring '"'
list      ::= '[' keyvalues ']'
sign      ::= '+' | '-'
digit     ::= ['0'-'9']
mantissa  ::= ('E' | 'e') sign? digit+
instring  ::= ASCII-{'&', '"'} | '&' ['a'-'z''A'-'Z']  ';'
\end{verbatim}
\endGML

\noindent The \GML{}-parser in \GGL{} can parse any well formed
\GML{} file that conforms to the above \texttt{BNF} grammar
specification. However the parser interprets only a subset of ``known''
key--value pairs (see according section) all other well-formed
key--value pairs are \textbf{silently ignored} (Note: a source of errors
is \underline{misspelling of known keys} since the parsing is
\underline{case-sensitive}).

